%!TEX root = ../main.tex
\addchap{Abstract}\label{chap:abstract} % basically chapter*, but keeps the chapter in the table of contents
Finding sets of optimal parameters for communication systems with complex or unknown channel models is a problem which can be approached thanks to \cgls{ml}. These parameters refer to the location and probability of occurrence of the constellation points. Together with autoencoders, pioneered in \citep{O'Shea}, which optimize together transmitter and receptor; and the appropriate loss function, which must maximize the channel's mutual information, this \cgls{ml}-based method has shown close-to-optimal results in \cite{Stark} and \cite{Aref} when performed over the \cgls{awgn} channel. Being the proposed methods different, we break-down and analyse the benefits and drawbacks to further understand their potential and applicability over more complex channels.
