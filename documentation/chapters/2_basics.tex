%!TEX root = ../main.tex
\chapter{preliminaries}\label{chap:preliminaries}
\section*{Classic Probabilistic Shaping}
\section*{Autoencoders}
\begin{table}[!htb]
    \centering
    \caption{List of simulation parameters.}
    \begin{tabular}{cc}
    Name & Value\\
    \hline
    Number of spans & \num{100}\\
    Number of transmitted symbols & \num{2.5e10}\\
    Transmission distance & \SI{1}{\km}
    \end{tabular}
    \label{tbl:demo}
\end{table}

You can use acronyms like this: \cgls{ber}, \cGls{awgn}, \acrlong{xt}, etc.
Not that the effect of the "c" in cgls is to only add acronyms if you actually use them more than once. So \cgls{ber} will get an acronym, while awgn won't.
