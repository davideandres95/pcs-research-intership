%!TEX root = ../main.tex
\makeglossaries % has to come first
%
% Example:
% \newacronym{xy}{XY}{Full Text}
% where xy is what you write in \[c]gls commands, XY is the acronym written in the text
% and "Full Text" is what is written out on the first occasion.
% The "c" prefix means "counting". It takes a bit of processing time, but it will figure out if an acronym is
% only used once and in that case only print the long version and not introduce the acronym at all.
%
% To use these acronyms in the text:
% \cgls{xy} % regular call, writes it out the first time, acronym otherwise
% \cglspl{xy} % same as \cgls, but plural form
% \Gls{xy} % same as \cgls, but capitalized
% \Glspl{xy} % same as \cglspl, but capitalized
% \acrshort{xy} same as \gls, but always prints the short version, even if it's the first time in the document
% \acrlong{xy} same as \gls, but always prints the long version
% \acrfull{xy} same as \gls, but always prints the long and short version
%
% Usually you don't include the glossary in a paper, but for a thesis you should.
% To do that:
% * put \makeglossaries in the preamble
% * call this after the first latex run: "cd build && makeglossaries main" and run latex again
% * add \printglossary where it should be
%
\newacronym{awgn}{AWGN}{additive white gaussian noise}
%
\newacronym{ber}{BER}{bit error rate}
%
\newacronym{xt}{XT}{crosstalk}
%
\newacronym{snr} {SNR} {signal-to-noise ratio}
%
\newacronym{ask} {ASK} {amplitude shift keying}
%
\newacronym{qam} {QAM} {quadrature amplitude modulation}
%
\newacronym{dl} {DL} {deep learning}
%
\newacronym{nn} {NN} {neural network}
%
\newacronym{sgd} {SGD} {stochastic gradient descent}
%
\newacronym{mi} {MI} {mutual information}